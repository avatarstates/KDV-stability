%% This is a LaTeX template for preparing papers for Publ. Inst. Math.; version January 2016
%% Please delete everything begining with %% (DOUBLE %).

% Submission number: 
\documentclass[a4paper,draft]{amsproc}
\usepackage{amssymb}
\usepackage[hyphens]{url} \urlstyle{same}
%\usepackage[dvips]{graphicx} %% Package for inserting illustrations/figures
\usepackage{amsthm}% proof can be used
%% The following packages are useful (you may want to use them):
%\usepackage{refcheck} %% Checks whether enumerated equations are referred to or not.
                       %% Please remove unnecessary numbers.
%\usepackage{cmdtrack} %% Checks whether all author defined macros are used or not
                       %% (see the end of .log file); unused ones should be removed.
%% Both packages have limitations---consult the package documentation.

\theoremstyle{plain}
 \newtheorem{thm}{Theorem}[section]
 \newtheorem{prop}{Proposition}[section]
 \newtheorem{lem}{Lemma}[section]
 \newtheorem{cor}{Corollary}[section]
\theoremstyle{definition}
 \newtheorem{exm}{Example}[section]
 \newtheorem{dfn}{Definition}[section]
\theoremstyle{remark}
 \newtheorem{rem}{Remark}[section]
 \numberwithin{equation}{section}

%% Please, do not change the following four lines:
\renewcommand{\le}{\leqslant}\renewcommand{\leq}{\leqslant}
\renewcommand{\ge}{\geqslant}\renewcommand{\geq}{\geqslant}
\renewcommand{\setminus}{\smallsetminus}
\setlength{\textwidth}{28cc} \setlength{\textheight}{42cc}

\title[Running title / header]{None}

\subjclass[2010]{Primary REQUIRED; Secondary OPTIONAL}

%% Please use the newest classification -- 2010
%% available at  http://msc2010.org/MSC-2010-server.html
%% and the newest amsproc.cls.
%% Please, classify to the third level,
%% e.g., 26A and 26Axx are not satisfsctory.

\keywords{optional, but desirable}

\author[Hongru Zhao]{\bfseries Hongru Zhao} %% Please write ful names, avoid initials

\address{ %% Put here your affiliation; street address is not required
Department of Mathematics \\ % \hfill (Received 00 00 201?)\\
Our University   \\ %\hfill (Revised  00 00 201?)\\
Town\\
Country}
\email{user@server}

%% OTHER AUTHOR(S):
%\author[]{}
%\address{ }
%\email{}

\thanks{Supported by ... } %% optional
\thanks{Communicated by ...} %% This will be filled in the journal office.

\begin{document}

{\begin{flushleft}\baselineskip9pt\scriptsize
%PUBLICATIONS DE L'INSTITUT MATH\'EMATIQUE\newline
%Nouvelle s\'erie, tome ??(1??)) (201?), od--do \hfill DOI: \\
MANUSCRIPT
\end{flushleft}}
\vspace{18mm} \setcounter{page}{1} \thispagestyle{empty}


\begin{abstract}
An abstract is REQUIRED!
Please do not use author defined macros in the abstract
and avoid references to anything in the paper,
since the abstract will be detached from the article.
\end{abstract}

\maketitle

\section{Section title}  %% Please avoid complex formulas in (sub)titles

This template contains detailed instructions for preparing manuscripts for our journal.
Missing to follow them may cause rejecting of the manuscript without further processing.
So, please read both the source and compiled texts.

Insert your text here. Make sure that it is written in correct English. 
The guides \cite{TrzD,TrzE} may help you with it.

If there are subsections, then you may use


\section{Proof for Exponential stability of Semigroup}  %% Please avoid complex formulas in (sub)titles
\subsection{Asymptotic behavior of the roots of the characteristic polynomial}
\begin{lem} \label{key_lemma}
Let $q_s(\lambda)= \lambda^3-is\lambda^2+\lambda+is$ be the characteristic polynomial of the $3^{rd}$ order ODE, with root ${\lambda_j=\lambda_j(s)},1\leq j\leq 3$ and let $p_s(\lambda)= (\lambda-\mu_1)(\lambda-\mu_2)(\lambda-\mu_3)$ be another polynomial, where $\mu_1=1-\frac{i}{s},\mu_2=-1-\frac{i}{s}, \mu_3=i(s+\frac{2}{s})$. 
Then for any $ K>0, \eta \in$ $(0,2)$, there exist $M>0$ such that:  
$|\lambda_1-\mu_1|\leq \frac{K}{|s|^{2-\eta}}$, 
$|\lambda_2-\mu_2|\leq \frac{K}{|s|^{2-\eta}}$ and $|
\lambda_3-\mu_3|\leq \frac{K}{|s|^{3-\eta}}$, when $ |s|>M$.
\end{lem}
\begin{proof}
Fixed  $K>0, \eta \in$  n$(0,2)$, we let $\varepsilon_1=\varepsilon_2=\frac{K}{|s|^{2-\eta}},\varepsilon_3=\frac{K}{|s|^{3-\eta}}$ and let $C_j:=\{z\in\mathbb{C} :|z-\mu_j|=\varepsilon_j\},1\leq j\leq 3$. \\
Now we are going to find a lower bound of $p_s=\lambda ^3-i s \lambda ^2+\left(1+\frac{3}{s^2}\right) \lambda+(\frac{2 i}{s^3}+\frac{3 i}{s}+i s )$ on each circle. Let $\mu_j+h\in C_j$. Then if $|s|$ is large enough, the following inequalities hold.
\begin{displaymath}
|p_s(\mu_1+h)|=|{h (1+h) (2+h)-\frac{3 i h (2+h)}{s}-i h (2+h) s}|\geq \dfrac{K}{|s|^{1-\eta}}=:v_1\\
\end{displaymath}
\begin{displaymath}
|p_s(\mu_2+h)|=|{(-2+h) (-1+h) h-\frac{3 i (-2+h) h}{s}-i (-2+h) h s}|\geq \dfrac{K}{|s|^{1-\eta}}=:v_2\\
\end{displaymath}
\begin{displaymath}
|p_s(\mu_3+h)|=|{h \left(-7+h^2\right)-\frac{9 h}{s^2}+\frac{6 i h^2}{s}+2 i h^2 s-h s^2}|\geq \dfrac{K}{2|s|^{1-\eta}}=:v_3.\\
\end{displaymath}
For any $z_j\in C_j$,  $|z_1|\leq 2,|z_2|\leq 2,|z_3|\leq |s|+1$ .\\
Then there exist $M>0$, such that when $|s|>M$, 
\begin{displaymath}
|q_s(z_j)-p_s(z_j)|\leq\dfrac{3}{s^2}|z_j|+\dfrac{4}{|s|}<v_j\leq |p_s(z_j)|.
\end{displaymath}
Therefore, it follows by Rouche’s Theorem that $(q_s(\lambda)-p_s(\lambda))+p_s(\lambda)=q_s(\lambda)$ and  $p_s(\lambda)$ have the same number of zeros, counting multiplicities in the interior of $C_j$, i.e.  $|
\lambda_j-\mu_j|\leq \varepsilon_j,j=1,2,3$.
\end{proof}
Since we have already proved that $A$ is a generator of a contraction semigroup and if positive number $L \not\in {\mathcal{L}}=\{L>0:-100+20 \left(k^2-k l+l^2\right) \alpha^2 +4 \left(k^2-k l+l^2\right)^2 \alpha ^4+k^2 (k-l)^2 l^2 \alpha ^6=0,\alpha =\frac{2\pi}{L},k>l>0, k,l \in \mathbb{N^+}\}$, then $(is-A)^{-1}$ exist for all $s\in {\mathbb{R}}$.\\
Hence the only condition we need to verify for the exponential stability is that $sup_{s\in {\mathbb{R}}}||(is-A)^{-1}||$ is bounded.\\
\linebreak
\begin{thm}
	If positive number $L \not\in {\mathcal{L}}$, then the semigroup is exponential stable. 
\end{thm}
\begin{proof}
Suppose $||(is-A)^{-1}||$ is unbounded, which means we can find $f_n\in H^1_0$, $y_n \in \mathcal{D(A)}$, real sequence ${s_n}$, such that $(is_n-A)y_n=f_n$, $||y_n||_{H}=1$ ,$||f_n||_{H}\rightarrow 0$ and $s_n\rightarrow \infty$.\\
First of all we are going to give an integral formula for $(is-A)^{-1}$, i.e. solve for the following boundary value problem for third order ODE. 
\begin{equation*}
\left\{
\begin{array}{lr}
y'''-isy''+y'+isy=f-f''  &  \\
y(0)=y(L)=y'(L)=0
\end{array}
\right.
\end{equation*}
The solution is $y(t)=y_p(t)+y_h(t)$, where $y_p(t)$ is a particular solution and $y_h(t)$ is the solution to the homogeneous differential equation. 

To make the notation simpler, we omit the subscript $n$. We consider $f''\in H^{-1}$. \\
Let $q_s(\lambda)= \lambda^3-is\lambda^2+\lambda+is$ to be characteristics polynomial. Since when $s$ is big enough, by lemma \ref{key_lemma},  $q_s$ has no multiple roots, namely $\lambda_1,\ \lambda_2$ and $\lambda_3$.\\
We will use cyclic notation for subscript, for example $\lambda_{j+3}=\lambda_j$.
 Let the fundamental solutions to be $y_i(x)=e^{\lambda_i (x-L)}$, Wronskian to be $W(x)=(\lambda_3-\lambda_2)(\lambda_2-\lambda_1)(\lambda_3-\lambda_1)e^{is(x-L)}$,and $W_i(x)=y_{i+1}(x)y'_{i+2}(x)-y'_{i+1}(x)y_{i+2}(x)$.
Then we can write down a formally particular solution by variation of parameters, namely:\\
\begin{equation*} 
	\begin{split}
		&y^0_p(t)\\=&y_1(x)\int_{L}^{x}\frac{W_1(t)f(t)}{W(t)} d t+y_2(x)\int_{L}^{x}\frac{W_2(t)f(t)}{W(t)} d t+y_3(x)\int_{L}^{x}\frac{W_3(t)f(t)}{W(t)} d t-\\&y_1(x)\int_{L}^{x}\frac{W_1(t)}{W(t)}f''(t) d t-y_2(x)\int_{L}^{x}\frac{W_2(t)}{W(t)}f''(t) d t-y_3(x)\int_{L}^{x}\frac{W_3(t)}{W(t)}f''(t) d t\\
		=&y_1(x)\int_{L}^{x}\frac{W_1(t)f(t)}{W(t)} d t+y_2(x)\int_{L}^{x}\frac{W_2(t)f(t)}{W(t)} d t+y_3(x)\int_{L}^{x}\frac{W_3(t)f(t)}{W(t)} d t+\\&y_1(x)\int_{L}^{x}(\frac{W_1(t)}{W(t)})'f'(t) d t+y_2(x)\int_{L}^{x}(\frac{W_2(t)}{W(t)})'f'(t) d t+y_3(x)\int_{L}^{x}(\frac{W_3(t)}{W(t)})'f'(t) d t
		-\\&y_1(x)\frac{W_1(t)f'(t)}{W(t)} \bigg|_L^x 
		-y_2(x)\frac{W_2(t)f'(t)}{W(t)} \bigg|_L^x
		-y_3(x)\frac{W_3(t)f'(t)}{W(t)} \bigg|_L^x\\
		=&y_1(x)\int_{L}^{x}\frac{W_1(t)f(t)}{W(t)} d t+y_2(x)\int_{L}^{x}\frac{W_2(t)f(t)}{W(t)} d t+y_3(x)\int_{L}^{x}\frac{W_3(t)f(t)}{W(t)} d t+\\&y_1(x)\int_{L}^{x}(\frac{W_1(t)}{W(t)})'f'(t) d t+y_2(x)\int_{L}^{x}(\frac{W_2(t)}{W(t)})'f'(t) d t+y_3(x)\int_{L}^{x}(\frac{W_3(t)}{W(t)})'f'(t) d t
		+\\&y_1(x)\frac{W_1(L)f'(L)}{W(L)}
		+y_2(x)\frac{W_2(L)f'(L)}{W(L)} 
		+y_3(x)\frac{W_3(L)f'(L)}{W(L)} 
	\end{split}
\end{equation*}
Since we do not have control on $f''$, we integrate by part to cancel the $f''$ term. Notice the fact that the coefficient of $f'(x)$ is zero, since $y_1(x)W_1(x)+y_2(x)W_2(x)+y_3(x)W_3(x)=0$, which comes from the determinant definition of Wronskian and Laplace expansion, and the fact that $y_1(x)\frac{W_1(L)f'(L)}{W(L)}+y_2(x)\frac{W_2(L)f'(L)}{W(L)} +y_3(x)\frac{W_3(L)f'(L)}{W(L)}$ is a solutions to the homogeneous differential equation. Hence we obtain another particular solution as following.
\begin{equation*}
\begin{split}
y_p(t)&=y_1(x)\int_{L}^{x}\frac{W_1(t)f(t)}{W(t)} d t+y_2(x)\int_{L}^{x}\frac{W_2(t)f(t)}{W(t)} d t+y_3(x)\int_{L}^{x}\frac{W_3(t)f(t)}{W(t)} d t+\\&y_1(x)\int_{L}^{x}(\frac{W_1(t)}{W(t)})'f'(t) d t+y_2(x)\int_{L}^{x}(\frac{W_2(t)}{W(t)})'f'(t) d t+y_3(x)\int_{L}^{x}(\frac{W_3(t)}{W(t)})'f'(t) d t
\end{split}
\end{equation*}
One can easily verify that this new particular solution satisfies $\langle \phi, q_s(D)y_p\rangle=\langle \phi,f-f''\rangle$ in the sense of distribution, where for all $\phi \in C^{\infty}$ and for all $\phi \in H_0^1(0,L)$.\\
Now let's consider homogeneous equation with boundary condition. 
\begin{equation*}
\left\{
\begin{array}{lr}
y_h'''-isy_h''+y'+isy_h=0  &  \\
y_h(0)+y_p(0)=0,y_h(L)+y_p(L)=0,y_h'(L)+y_p'(L)=0
\end{array}
\right.
\end{equation*}
Let $y_h(x)=c_1y_1(x)+c_2y_2(x)+c_3y_3(x)$.\\
On the one hand it is easy to see that $y_p(L)=0$, on the other hand $y_p(L)=0$ comes from $y_1(x)W_1(x)+y_2(x)W_2(x)+y_3(x)W_3(x)=0$, and
\begin{equation*}
\begin{split}
&y_1(x)(\frac{W_1(x)}{W(x)})'+y_2(x)(\frac{W_2(x)}{W(x)})'+y_3(x)(\frac{W_3(x)}{W(x)})'\\
=&(y_1(x)\frac{W_1(x)}{W(x)}+y_2(x)\frac{W_2(x)}{W(x)}+y_3(x)\frac{W_3(x)}{W(x)})'
-\frac{y_1'(x)W_1(x)+y_2'(x)W_2(x)+y_3'(x)W_3(x)}{W(x)}\\
=&(\dfrac{W(x)}{W(x)})'-\frac{0}{W(x)}=0
\end{split}
\end{equation*}
Then we could set up a linear equation. 
\begin{equation*} 
\begin{split}
\begin{bmatrix} 1 & 1 &1 \\ \lambda_1 & \lambda_2 & \lambda_3 \\ e^{-\lambda_1 L}&e^{-\lambda_2L}&e^{-\lambda_3 L} \end{bmatrix} 
\begin{bmatrix} c_1\\c_2\\c_3 \end{bmatrix} =\begin{bmatrix} 0\\0\\-y_p(0) \end{bmatrix}
\end{split}
\end{equation*}
Let $D_s$ represents the determinant of this matrix. 
Then
\begin{equation*}
D_s= (e^{-L \lambda _2}-e^{-L \lambda _3}) \lambda _1+(e^{-L \lambda _3}-e^{-L \lambda _1}) \lambda _2+(e^{-L \lambda_1}-e^{-L \lambda _2}) \lambda _3
\end{equation*}
Using lemma, we know that \\
\begin{equation*} 
\begin{split}
 |D_s| &\geq |(e^{-L \lambda_1}-e^{-L \lambda _2}) \lambda _3|-| (e^{-L \lambda _2}-e^{-L \lambda _3}) \lambda _1|-|(e^{-L \lambda _3}-e^{-L \lambda _1}) \lambda _2|\\
 &\geq \dfrac{e^L-e^{-L}}{2}|s|-4(e^L+e^{-L})>0,\\
\end{split}
\end{equation*}
 when s is big enough. Hence when $n$ is big enough, the solution of homogeneous equation with such boundary condition always exist and unique. Now we can solve for the coefficient, namely:\\
 \begin{displaymath}
c_1=\frac{(\lambda_2-\lambda_3)y_p(0)}{Ds},c_2=\frac{(\lambda_3-\lambda_1)y_p(0)}{Ds},c_3=\frac{(\lambda_1-\lambda_2)y_p(0)}{Ds}.
\end{displaymath}
Then we get $(is-A)^{-1}[f](x)=y_h(x)+y_p(x)$ is the solution for ODE.\\
Secondly, we will write down the $y_p(x)$ again, trying to find uniform bound for it.
\begin{displaymath}
\begin{aligned}
y_p(x)=&e^{(-L+x) \lambda
		_1} \int_L^x \frac{e^{(-L+t) \left(-i s+\lambda _2+\lambda _3\right)} f(t)}{\left(-\lambda _1+\lambda _2\right) \left(\lambda _1-\lambda _3\right)}\, dt+\\
&{e^{(-L+x) \lambda _2} \int_L^x \frac{e^{(-L+t) \left(-i s+\lambda _1+\lambda _3\right)} f(t)}{\left(\lambda _1-\lambda _2\right) \left(\lambda
		_2-\lambda _3\right)} \, dt+}\\
	&{e^{(-L+x) \lambda _3} \int_L^x \frac{e^{(-L+t) \left(-i s+\lambda _1+\lambda _2\right)} f(t)}{\left(\lambda _1-\lambda
		_3\right) \left(-\lambda _2+\lambda _3\right)} \, dt+}\\
&{e^{(-L+x) \lambda _3} \int_L^x -\frac{e^{(-L+t) \left(-i s+\lambda _1+\lambda _2\right)} \left(-i s+\lambda _1+\lambda _2\right) f'(t)}{\left(\lambda
		_1-\lambda _3\right) \left(\lambda _2-\lambda _3\right)} \, dt+}\\
&{e^{(-L+x) \lambda _2} \int_L^x -\frac{e^{(-L+t) \left(-i s+\lambda _1+\lambda _3\right)} \left(-i s+\lambda _1+\lambda _3\right) f'(t)}{\left(\lambda
		_1-\lambda _2\right) \left(-\lambda _2+\lambda _3\right)} \, dt+}\\
&{e^{(-L+x) \lambda _1} \int_L^x -\frac{e^{(-L+t) \left(-i s+\lambda _2+\lambda _3\right)} \left(-i s+\lambda _2+\lambda _3\right) f'(t)}{\left(\lambda
		_1-\lambda _2\right) \left(\lambda _1-\lambda _3\right)} \, dt}.
\end{aligned}
\end{displaymath}
Using lemma \ref{key_lemma} again, we know $\lambda _1,\lambda _2$, Re$ \lambda _3$ are bounded , $\lambda _3-{is}\rightarrow0$ and $|\lambda _1-\lambda _2|>1$, when $s$ is large enough. Detailedly, $e^{(-L+x) \lambda_i}$, $e^{(-L+t) (is-\lambda_i-\lambda_{i+1})}$, $|-i s+\lambda_{1}+\lambda_{3}|$, $|-i s+\lambda_{2}+\lambda_{3}|$ and $|\frac{1}{\lambda_2-\lambda_1}|$ are bounded uniformly respect to large enough $s$$ $  $ $, as well as $|{\lambda_3-\lambda_1}|\sim |{\lambda_3-\lambda_2}|\sim|{-is+\lambda_1+\lambda_2}|\sim |s|$ have the same order.
Then we know the following inequality holds, by simply using Hölder's inequality.
\begin{equation}\label{yp-bounded}
	|s\cdot  y_p(x)|\leq M ||f||_H
\end{equation}
\begin{equation}\label{yp-equi}
	|y_p(x)-y_p(y)|\leq M_s |x-y| \cdot  ||f||_H 
\end{equation}
\begin{equation}\label{yp-norm}
	||y_p||_{L_2}\leq M ||f||_H  
\end{equation}
where M is a constant independent of $s$ and $x$, $M_s$ is a constant maybe depend on $s$ but independent of $x$. (\ref{yp-equi}) holds for fixed $s$ since the product of two lipschitz and uniformly bounded functions is still lipschitz and uniformly bounded.\\
A special case is that $|y_p(0)|\leq M ||f||_H$, which lead to\\ 
\begin{displaymath}
\begin{aligned}
&|c_j|\leq \frac{M  ||f||_H}{|s|},\\
&|\lambda_jc_j|\leq \frac{M  ||f||_H}{|s|},
\end{aligned}
\end{displaymath}
by using lemma \ref{key_lemma} and the lower bound of $|D_s|$.  \\
Hence 
\begin{equation}\label{yh-bounded}
|y_h(x)|\leq\dfrac{ M ||f||_H}{|s|}
\end{equation}
\begin{equation}\label{yh'-bounded}
|y'_h(x)|\leq \dfrac{ M ||f||_H}{|s|}
\end{equation}
\begin{equation}\label{yh-norm}
||y_h||_{H}\leq \dfrac{ M ||f||_H}{|s|}.
\end{equation}
Next let's write down $y'_p(x)$, 
\begin{displaymath}
\begin{aligned}
y'_p(x)=&\lambda
_1e^{(-L+x) \lambda
	_1} \int_L^x \frac{e^{(-L+t) \left(-i s+\lambda _2+\lambda _3\right)} f(t)}{\left(-\lambda _1+\lambda _2\right) \left(\lambda _1-\lambda _3\right)}\, dt+\\
&\lambda
_2{e^{(-L+x) \lambda _2} \int_L^x \frac{e^{(-L+t) \left(-i s+\lambda _1+\lambda _3\right)} f(t)}{\left(\lambda _1-\lambda _2\right) \left(\lambda
		_2-\lambda _3\right)} \, dt+}\\
&\lambda
_3{e^{(-L+x) \lambda _3} \int_L^x \frac{e^{(-L+t) \left(-i s+\lambda _1+\lambda _2\right)} f(t)}{\left(\lambda _1-\lambda
		_3\right) \left(-\lambda _2+\lambda _3\right)} \, dt+}\\
&\lambda
_3{e^{(-L+x) \lambda _3} \int_L^x -\frac{e^{(-L+t) \left(-i s+\lambda _1+\lambda _2\right)} \left(-i s+\lambda _1+\lambda _2\right) f'(t)}{\left(\lambda
		_1-\lambda _3\right) \left(\lambda _2-\lambda _3\right)} \, dt+}\\
&\lambda
_2{e^{(-L+x) \lambda _2} \int_L^x -\frac{e^{(-L+t) \left(-i s+\lambda _1+\lambda _3\right)} \left(-i s+\lambda _1+\lambda _3\right) f'(t)}{\left(\lambda
		_1-\lambda _2\right) \left(-\lambda _2+\lambda _3\right)} \, dt+}\\
&\lambda
_1{e^{(-L+x) \lambda _1} \int_L^x -\frac{e^{(-L+t) \left(-i s+\lambda _2+\lambda _3\right)} \left(-i s+\lambda _2+\lambda _3\right) f'(t)}{\left(\lambda
		_1-\lambda _2\right) \left(\lambda _1-\lambda _3\right)} \, dt}.
\end{aligned}
\end{displaymath}
Using the same skill, we know that 
\begin{equation}\label{yp'-bounded}
|y'_p(x)|\leq M ||f||_H
\end{equation}
\begin{equation}\label{yp'-equi}
|y'_p(x)-y'_p(y)|\leq M_s |x-y| \cdot  ||f||_H
\end{equation}
\begin{equation}\label{yp'-norm}
||y'_p||_{L_2}\leq M ||f||_H.
\end{equation}
Now we combine our conclusions (\ref{yp-norm}),(\ref{yh-norm}),(\ref{yp'-norm}) and assumption $||f_n||_{H}\rightarrow 0$, we obtain $||y_n||_{H}\leq M ||f_n||_H\rightarrow0$. However, $||y_n||_H=1$ is a contradiction. \\
Using theorem, we obtain exponential stability of Semigroup.
\end{proof}
\begin{rem}
 When $s=0$, we can solve for determinant $D_0=2i(cos(L)-1)\neq 0$, and get $L\neq 2k\pi ,k\in \mathbb{N^+}$, since $y, y'$ are both uniformly bounded (\ref{yp-bounded}),(\ref{yh-bounded}),(\ref{yh'-bounded}),(\ref{yp'-bounded}) and  equicontinuous (\ref{yp-equi}),(\ref{yp'-equi}), by Arzelà–Ascoli theorem, we can find subsequence convergent uniformly, which implies $A^{-1}$ is compact operator.
\end{rem}
\begin{rem}
If positive number $L \not\in \cup_{k>l>0}[\dfrac{2\pi\sqrt{k^2-kl+l^2}}{\sqrt{c}},\dfrac{2\pi\sqrt{k^2-kl+l^2}}{\sqrt{3}}]$, where $c=\dfrac{5(\sqrt{5}-1)}{2}\approx 3.09017$, then the semigroup is exponential stable.
\end{rem}
\begin{figure}[htb]
%\includegraphisc[width=99mm]{filename.eps}
\caption{}
\label{some label}
\end{figure}

Besides the standard handbooks on \LaTeX\ \cite{Gr,Lmp,Lbible},
please consult the short and useful guide \cite{TrzG}.


\section{List of references}

The list of references should be written as below.

Only \emph{standard} abbreviations for names of journals and other serials
should be used (see \url{http://zbmath.org/journals/}).


\bibliographystyle{amsplain}
\begin{thebibliography}{n} %% n is number of items, or the largest label

\bibitem{Exm_paper} A.\,U. Thor, (not Thor, A.U.!)
\emph{Title of paper},
J. Math. \textbf{99} (2008), 111--222.

\bibitem{Exm_in_book} A.\,U. Thor,
\emph{Title of paper},
in: E. Ditor (ed.), \emph{Title of Book}, Publisher, City, Year, 888--999.

\bibitem{Gr} G. Gr\"atzer,
\emph{More Math Into \LaTeX}, 4th ed.,
Springer, 2007.

\bibitem{Lmp} L. Lamport,
\emph{\LaTeX: A Document Preparation System}, 2nd ed.,
Addison-Wesley, 1994.

\bibitem{Lbible} F. Mittelbach,  M. Goossens (with J. Braams, D. Carlisle, C. Rowley),
\emph{The \LaTeX\ Companion}, 2nd ed.,
Addison-Wesley, 2004.

\bibitem{TrzG} J. Trzeciak, \emph{Writing mathematical papers---a few tips}, available at\\
\url{https://www.impan.pl/wydawnictwa/dla-autorow/writing.pdf}

\bibitem{TrzE} J. Trzeciak, \emph{Writing Mathematical Papers in English},
European Mathematical Society, Zurich, 2005; a version available at
\url{https://libgen.unblocked.zone/book/1043688}

\bibitem{TrzD} J. Trzeciak, \emph{Mathematical English Usage. A Dictionary}, available at\\
\url{http://www.impan.pl/Dictionary}

\end{thebibliography}

\end{document}

%% To be filled in the journal office:

@author:
@affiliation:
@title:
@language: English
@pages:
@classification1:
@classification2:
@keywords:
@abstract:
@filename:
@EOI


